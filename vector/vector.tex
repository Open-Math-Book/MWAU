\chapter{Vectors and Transformations}\label{vect_n_trans}

\setcounter{section}{-1}
% New Section %%%%%%%%%%%%%%%%%%%%%%%%%%%%%%%%%%%%%%%%%%%%%%%%
\section{Mathematical Outcome}\label{sec:VectorOutcome}
%%%%%%%%%%%%%%%%%%%%%%%%%%%%%%%%%%%%%%%%%%%%%%%%%%%%%%%%%%%%%%

% New Section %%%%%%%%%%%%%%%%%%%%%%%%%%%%%%%%%%%%%%%%%%%%%%%%
\section{What is the matrix?}\label{sec:Matrix}
%%%%%%%%%%%%%%%%%%%%%%%%%%%%%%%%%%%%%%%%%%%%%%%%%%%%%%%%%%%%%%

\newpage
%%%%%%%%%%%%%%%%%%%%%%%%%%%%%%%%%%%%%%%%%%%%%%%%%%%%%%%%%%%%%%
\subsection{Entrance Activity: Vectors and Scalars}

In Engineering and Physics they are very careful to distinguish between two basic types of quantities: vectors and scalars.  Scalar quantities are what we're probably most used to -- they're physical things that can be represented with a single number.

\begin{itemize}
    \item Janet is $145$ cm tall.
    \item It's $61$ degrees Fahrenheit outside today.
    \item Pouring that slab will take 35 cubic yards of concrete.
\end{itemize}

Vector quantities can be thought of in two ways.  Most Scientists will automatically say a vector ``has a magnitude {\em and} a direction.''  Another way to think of it is that a vector will require two or more numbers to represent it.

\begin{itemize}
    \item The wind today is 15 mph out of the northwest.
    \item My mother's house is 3 blocks north and 2 blocks east of mine. 
    \item The boat is heading due south at 18 mph, but the tide is also carrying it to the east at 2 mph.
\end{itemize}

Question: Is weight a vector or a scalar?  (It might help to visualize two people weighing themselves simultaneously -- one in New Zealand and one in Connecticut.)

\vspace{.75in}

Question: Come up with as many examples of physical quantities as you can that are vectors and scalars
\bigskip

\centerline{\begin{tabular}{|c|c|} \hline
 \bigstrut scalars &  \bigstrut vectors \\ \hline
\rule{200pt}{0pt} \rule{0pt}{150pt} & \rule{200pt}{0pt} \rule{0pt}{150pt} \\ \hline
\end{tabular}}
\bigskip

\wbnewpage

\subsection{Activity: A Grocery List}

Suppose you want to make fresh salsa for a party.  You might have a grocery list that included onions, tomatoes, jalapenos and avocados.  The quantities of each that you intend to buy form a list of numbers -- that's a vector!  The store you go to has a different list of numbers -- how much does each type of produce cost?  

Suppose your recipe calls for 1 onion, 4 tomatoes, 3 jalapenos and 2 avocados.  The corresponding vector is

\[ \left(1, 4, 3, 2\right). \]

Further suppose that the store's price vector is

\[ \left(\$0.49, \$0.99, \$0.78, \$1.25 \right). \]

Question: What computation to you do with these two lists to find out how much money your salsa is going to cost?

\wbvfill

What we just saw is an example of using two vectors to produce a scalar.  It also illustrates a fairly typical situation: one of the vectors is fixed and the other is variable.

Question:  Which is the fixed vector and which is the variable vector?  Does the store change its prices from one customer to the next, or do different customers follow slightly different salsa recipes?

\wbvfill

