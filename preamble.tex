\usepackage{fullpage}
\usepackage{times}
\usepackage{amsmath} 
\usepackage{latexsym}
\usepackage{wrapfig}
\usepackage{amssymb,amsfonts,amsthm}
\usepackage{graphicx}
\usepackage{graphics}
\usepackage{url}
\usepackage{color}
\usepackage{float}
\usepackage{multirow}
\usepackage{multicol}
\usepackage{makeidx}
\usepackage{nextpage}
\usepackage[figuresright]{rotating}
\usepackage{appendix}
\usepackage[T1]{fontenc}
\usepackage[nottoc]{tocbibind} % Table of contents
\usepackage[titles,subfigure]{tocloft}
\usepackage{subfig}
\usepackage{colortbl}
\usepackage{ wasysym }
\usepackage{rotating}
\usepackage[table]{xcolor}
\usepackage[hyperfootnotes=false]{hyperref}
\hypersetup{
    colorlinks=true,
    linkcolor=blue}
\usepackage{cancel}
\let\checkmark\relax
% \usepackage{dingbat}
\usepackage{ifthen}
\usepackage{longtable}
\usepackage{enumitem}

\newcounter{bluefigure}
\newcounter{bluetable}
\makeatletter
\renewcommand\theenumii{(\alph{enumii})}
\renewcommand\labelenumii{(\alph{enumii})}
\renewcommand\p@enumiii{\theenumi\theenumii}
\makeatother

    \makeatletter
    \renewcommand\part{%
      \if@openright
        \cleardoublepage
      \else
        \clearpage
      \fi
      \thispagestyle{empty}%
      \if@twocolumn
        \onecolumn
        \@tempswatrue
      \else
        \@tempswafalse
      \fi
      \null\vfil
      \secdef\@part\@spart}
    \makeatother 

\definecolor{burntorange}{RGB}{207, 83, 0}
\definecolor{lightblue}{rgb}{0.93,0.95,1.0}
\definecolor{lightgray}{gray}{0.9}

\def\censor#1{\censorloopword#1 \nil}
\def\censorloopword#1 #2\nil{%
  \phantom{#1} % <- Note the space!
  \ifx&#2&% #2 is empty, then & equals &
    \let\next\relax
  \else
    \def\next{\censorloopword#2\nil}% iterate
  \fi
  \next\ignorespaces}

\def\do#1{#1}

\newtheorem{theorem}{Theorem}[chapter]  % [chapter] was needed to reset the numbers of the theorems after a chapter
\newtheorem*{question}{Question}
\newtheorem{act}{Activity}[chapter]
\renewenvironment{proof}{\noindent{\em Proof}:}{\hfill \maltese}
\newenvironment{remark}{\medskip\noindent{\bf Remark}.}{}
\newenvironment{references}{\begin{thebibliography}{99}}{\end{thebibliography}}
\newcommand{\Strut}{\rule[0ex]{0ex}{4ex}}
\renewcommand{\thefootnote}{} % suppress footnote mark

\theoremstyle{definition} % Makes text upright and title upright and bold.
\newtheorem{definition}[theorem]{Definition}

  \newcounter{case}[theorem]
  \newcommand{\startcases}{\setcounter{case}{0}}
  \newcommand{\case}[1]{\refstepcounter{case}\bigskip\noindent{\bf Case \thecase:}%
  \quad#1\hfill \vspace{11pt}}
  %
  \newcounter{subcase}[case]
  \newcommand{\startsubcases}{\setcounter{subcase}{0}}
  \newcommand{\subcase}[1]{\refstepcounter{subcase}\bigskip\noindent{\bf Case \thecase.\thesubcase:}%
  \quad#1\hfill\vspace{11pt}}
  %
  \newcounter{subsubcase}[subcase]
  \newcommand{\startsubsubcases}{\setcounter{subsubcase}{0}}
  \newcommand{\subsubcase}[1]{\refstepcounter{subsubcase}\bigskip\noindent%
  {\bf Case \thecase.\thesubcase.\thesubsubcase:}%
  \quad#1\hfill\vspace{11pt}}
  %
  \newcounter{subsubsubcase}[subsubcase]
  \newcommand{\startsubsubsubcases}{\setcounter{subsubsubcase}{0}}
  \newcommand{\subsubsubcase}[1]{\refstepcounter{subsubsubcase}\bigskip\noindent%
  {\bf Case \thecase.\thesubcase.\thesubsubcase.\thesubsubsubcase:}%
  \quad#1\hfill\vspace{11pt}}
  %
  \newcounter{subsubsubsubcase}[subsubsubcase]
  \newcommand{\startsubsubsubsubcases}{\setcounter{subsubsubsubcase}{0}}
  \newcommand{\subsubsubsubcase}[1]{\refstepcounter{subsubsubsubcase}\bigskip\noindent%
  {\bf Case \thecase.\thesubcase.\thesubsubcase.\thesubsubsubcase.\thesubsubsubsubcase:}%
  \quad#1\hfill\vspace{11pt}}

% Makes a better itemize that is closer together between two items.
\newenvironment{packedItem}{
\begin{itemize}
  \setlength{\itemsep}{1pt}
  \setlength{\parskip}{0pt}
  \setlength{\parsep}{0pt}
}{\end{itemize}}

% Makes a better itemize that is closer together between two items.
\newenvironment{packedEnum}{
\begin{enumerate}
  \setlength{\itemsep}{1pt}
  \setlength{\parskip}{0pt}
  \setlength{\parsep}{0pt}
}{\end{enumerate}}

\def\A{{\cal A}}
\newcommand{\inst}{\em}  % Use this command for black and white printing
\newcommand{\ds}{\ensuremath{\displaystyle}}
\newcommand{\town}{Spoonerville }
\newcommand{\wrapup}{Wrap-up}
\newcommand{\order}{order }
\newcommand{\orders}{orders }
\newcommand{\orderper}{order.}
\newcommand{\ordersper}{orders.}
\renewcommand{\-}{\ensuremath{\text{--}}}
\newcommand{\wt}{\ensuremath{\text{\normalfont \tt wt}}}
\newcommand{\val}{\ensuremath{\text{\normalfont \tt val}}}

\renewcommand{\strut}{ \rule[-8pt]{0pt}{22pt} } %invisible space (usually used to make blank entries in a table a little taller.)
\newcommand{\bigstrut}{ \rule[-8pt]{0pt}{32pt} } %bigger invisible space
\newcommand{\hstrut}{ \rule{36pt}{0pt} } %horizontal invisible space

% make marginpar fit better
\setlength{\marginparwidth}{.75in}
\let\oldmarginpar\marginpar
\renewcommand\marginpar[1]{\-\oldmarginpar[\raggedleft\footnotesize #1]%
{\raggedright\footnotesize #1}}


\renewcommand{\cftsecfont}{\normalfont}
\renewcommand{\cftsecpagefont}{\normalfont}
\renewcommand{\cftsecleader}{\normalfont\cftdotfill{\cftsecdotsep}}

\renewcommand{\cftsecdotsep}{\cftdotsep}
\renewcommand{\cftbeforesecskip}{0em}


% Page-Format Details
\setlength{\oddsidemargin}{0.50in}     % Left side margin
\setlength{\evensidemargin}{0.0in}    % Right side margin
\setlength{\topmargin}{0.00in}         % Top margin
\setlength{\headheight}{0.00in}        % Header height
\setlength{\headsep}{0.00in}           % Separation between header and
                                       % main text
\setlength{\topskip}{0.00in}           % Top skip
\setlength{\textwidth}{6.00in}         % Width of the text
\setlength{\textheight}{8.61in}        % Height of the text

% Customizing Captions (Figures, Tables, etc)
\setlength{\floatsep}{0.15in}          % Space left between floats.
\setlength{\textfloatsep}{\floatsep}   % Space between last top float 
                                       % or first bottom float and the text
\setlength{\intextsep}{\floatsep}      % Space left on top and bottom 
                                       % of an in-text float 
\setlength{\abovecaptionskip}{0.10in}  % Space above caption
\setlength{\belowcaptionskip}{0.05in}  % Space below caption
\setlength{\captionmargin}{0.50in}     % Left/Right margin for caption
%\setlength{\captionwidth}{5.00in}      % Caption width
\setlength{\abovedisplayskip}{-0.10in} % Space before Math stuff
\setlength{\belowdisplayskip}{-0.10in} % Space after Math stuff


\makeindex

\renewcommand{\chaptermark}[1]{\markboth{#1}{}}

\usepackage{fancyhdr,graphicx,lastpage}%y

\fancyhf{} % clear the headers


\fancypagestyle{specialfooter}{%
  \fancyhf{}% Clear header/footer
  \renewcommand{\headrulewidth}{0pt}
  \fancyfoot[L]{Southern Connecticut State University}% Right header
  \fancyfoot[C]{\thepage}% Center footer
  \fancyfoot[R]{Mathematics Department}
}

\pagestyle{plain}% Set page style to plain.


\newcommand\blfootnote[1]{%
  \begingroup
  \renewcommand\thefootnote{}\footnote{#1}%
  \addtocounter{footnote}{-1}%
  \endgroup
}

\providecommand{\tabularnewline}{\\}

\newcommand{\activity}[1]{\noindent{}\textbf{Activity #1}}

%</preamble>

%<*IGP:intro>
%\title{\vspace{2.5in}
%Mathematics in the World Around Us\\
%Instructor's Guide}

\author{Leon Brin, Braxton Carrigan, Joseph Fields}


\date{\parbox{\linewidth}{\centering%
  \today\endgraf\bigskip
\endgraf
\vspace{2.5in}
}}

\newboolean{InWB}
\setboolean{InWB}{true}
\newboolean{InIG}
\setboolean{InIG}{false}
\newboolean{includehints}
\setboolean{includehints}{false}

\newcommand{\wbvfill}{\ifthenelse{\boolean{InWB}}{\vfill}{}}
\newcommand{\wbnewpage}{\ifthenelse{\boolean{InWB}}{\newpage}{}}

\newcommand{\Instr}[1]
{\ifthenelse{\boolean{InIG}}{\color{burntorange}\textbf{To the instructor:}#1\color{black}}{}}

\newcommand{\Hint}[1]{%
\ifthenelse{\boolean{includehints}}{{\bf Hint:}% \newline %
#1 }{ %don't print anything if yer not in the Instructor Guide
}% end of the ifthenelse
}% end of command definition

